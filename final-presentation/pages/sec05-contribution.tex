\section{贡献与价值}

\begin{frame}{贡献与价值}
    \begin{block}

        {\bf 1. 基线模型准确率高}
        \vskip 0.1em
        {\tiny 本文实现的行人重识别基线模型取得了接近原论文的准确率和识别性能,为后期进一步研究行人重识别领域其它问题提供了强有力的支持。}
        \vskip 0.8em

        {\bf 2. 多CPU集群的作用与潜力}
        \vskip 0.1em
        {\tiny 面向多CPU集群的深度行人重识别模型训练验证了CPU和GPU在进行海量单精度浮点数运算的性能差异,同时也展示了多CPU集群在训练深度神经网络方面的巨大潜力。}
        \vskip 0.8em

        {\bf 3. 高质量的行人重识别数据库}
        \vskip 0.1em
        {\tiny 本文推出的行人重识别数据集包含了17个摄像头,并且还包含了完整的行人跟踪场景,不仅在摄像头数量上远远多于现有主流的行人重识别数据集,而且为行人重识别问题新的评估方式的提出提供了基础。}
        \vskip 0.8em

        {\bf 4. 强化学习模型与人类认知相符}
        \vskip 0.1em
        {\tiny 本文提出的基于强化学习模型的摄像头部署方案选择模型,将强化学习算法运用到多摄像头选择问题当中,避免了穷举计算,且达到了理想的效果。}
    \end{block}
\end{frame}