\section{提问}

\begin{frame}{多CPU集群分布式训练与GPU的比较}
    \begin{block}

        单机GPU的训练速度为 {\bf 326} s~/~Epoch。按照已有的结果估计,大约需要包含20个节点的CPU集群才能追平GPU的速度。
        \vskip 1em

        由此可见在运算速度上,多CPU集群在单纯的计算速度上不占优势。而多CPU集群相对于GPU的优势在于:
        \vskip 1em

        {\bf 大内存 $\Rightarrow$ 大Batch Size $\Rightarrow$ 大学习率  $\Rightarrow$ 少训练批次}
    \end{block}
\end{frame}

\begin{frame}{复现成果与原文不一致的原因}
    \begin{block}

        \begin{enumerate}
            \item 模型参数的初始化方式不一致,导致我们的模型与原论文中的模型收敛到了不同的局部最小值;\\[0.8em]
            \item 不确定原始RPP模型中用于像素分类的神经网络层是否加入了Batch Normalization层和Dropout层,以及它们的具体参数是多少;\\[0.8em]
            \item 训练过程不一致,在原论文中没有清晰地说明三个训练阶段的Epoch数,以及每个训练阶段的学习率变化情况,所以只能靠人工调参。
        \end{enumerate}

    \end{block}
\end{frame}

\begin{frame}{未来可扩展的方向}
    \begin{block}

        \begin{enumerate}
            \item 行人重识别基线模型的模型结构以及超参数设置还有很大的调整空间,花费更多精力于此可以得到准确率更高的模型。\\[0.8em]
            \item 完整版本的多摄像头行人重识别数据库还有很多的数据预处理工作,以及在数据处理完成之后,在该数据集上还有更多可以研究的问题。\\[0.8em]
            \item 面向强化学习的摄像头部署方案选择模型,还可以往更深的网络结构探索,如何结合深度学习和强化学习的优势以应用于该特定问题,值得更加深入的研究。
        \end{enumerate}

    \end{block}
\end{frame}

\begin{frame}{在项目中的角色}
    \begin{block}

        项目中获得的帮助:

        \begin{enumerate}
            \item 指导老师提示使用强化学习进行多摄像头选择。\\[0.8em]
            \item 数据拍摄过程中,有5$\sim$7位同学帮忙召集演员、布置场地设备、记录时间。\\[0.8em]
        \end{enumerate}

        除上述工作外的剩余工作完全由作者个人完成。
    \end{block}
\end{frame}