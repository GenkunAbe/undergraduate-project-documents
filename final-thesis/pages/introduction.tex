\chapter{绪论}
本章首先阐述了论文的研究背景,说明论文的动机和主要观点。接着介绍了与论文主题密切相关的前人工作,分析评价其贡献与不足。然后形式化地对要解决的问题进行定义,明确研究工作的界限和规模。最后介绍论文的主要工作,以及论文的组织结构。

\section{研究背景}

随着视频监控技术的发展,无人值守的视频监控设备被越来越普遍地部署在国民社会的各个方面,在公安、交通、智能楼宇、金融、司法、教文卫等领域都有着不可替代的作用。具体的应用场景包括平安城市、卡口系统、工地监控、自助银行、监狱劳教和学前教育等。

在视频监控领域一个很重要且极具挑战性的问题是行人的重识别。行人重识别,指的是对多个视野不重叠的监控摄像头拍摄的同一个行人的图片进行匹配\cite{chen2018person},即重新识别一些之前出现过的行人。该工作的实现可以为人物搜索、特定人物跟踪等应用提供强有力的支持,进而应用在平安城市、工地监控、学前教育等场景。

行人重识别技术在实际应用中受诸多因素的影响,包括摄像头的部署位置、成像质量以及摄像头数量等等。面对一个从未部署过摄像头的监控场景,一个优秀的摄像头位置部署方案可以实现监控范围无死角、行人重识别准确率高以及跨摄像头持续跟踪的连续性强的效果。影响摄像头成像质量的因素有很多,包括摄像头自身的性能参数,摄像头部署角度不同导致的不同视野范围,以及摄像头的部署位置不同导致的不同光线条件。在理想情况下,摄像头的数量越多,得到的行人信息就越多,更有利于行人重识别算法的实施。但是在预算有限的情况下,摄像头的数量不可能无限增加。因此,如何选择摄像头的数量及其部署位置,使得行人重识别算法的性能最大化,便成为一个极具现实意义的研究问题。

要实现行人的重识别需要借助计算机视觉领域的技术。目前的计算机视觉算法的前沿关注点主要集中在深度学习技术。由于深度学习的海量计算需求和GPU强大的并行计算能力,大部分深度学习模型都是借助GPU来调整模型参数。然而使用GPU进行运算也存在许多问题,例如GPU设备通常比较昂贵,且少见于嵌入式设备中。同时GPU的显存普遍不高,限制了模型的规模以及每一批次数据的规模。与此同时,多CPU集群具有硬件成本低、搭建方式灵活以及部署广泛等优点,然而对于深度学习算法在多CPU集群上的研究与应用少之又少。因此,有必要对深度学习算法在多CPU集群上的性能表现进行深入的调查、分析和研究。

\section{研究现状}
本节介绍了与本课题密切相关的三个领域,行人重识别、多摄像头选择以及多CPU集群,其中的前人工作以及研究现状。

\subsection{行人重识别}

行人重识别的任务是将多个视野不重叠的监控摄像头所拍摄的同一行人进行匹配\cite{chen2018person}。与用于身份认证的人脸识别技术不同的是,行人重识别技术不是针对某个特定的行人进行建模,因为测试环节的数据可能没有在训练环节出现过。因此,行人重识别模型需要学习到一种判别两张图片空间特征和语义特征相似度的能力。要拥有该能力,则需要行人重识别模型具有强大的特征提取功能和距离度量功能。一个优秀的特征提取模型能够将行人图片中最具有判别性的特征很好地提取和表示,使得在所构造的特征空间内,不同行人所对应的特征的分界面尽可能地宽阔。一个优秀的距离度量模型能够使得相同行人的特征之间的距离尽可能近,不同行人的特征之间的距离尽可能远,即在特征空间内能够很好的将行人的特征进行正确地聚类。基于这样两个通用的建模阶段,在行人重识别领域研究的方法也主要分为两大类:基于表征学习(Representation Learning)的方法和基于度量学习(Metric Learning)的方法。

基于表征学习的方法将主要精力用于使模型拥有强有力的特征表示能力,使模型能够发现并表示行人图片中具有判别性的空间特征(如肤色、头发、衣着等)和语义特征(如性别、姿态、行为等)。得益于近年来深度学习\cite{simonyan2014very},尤其是深度卷积神经网络\cite{krizhevsky2012imagenet}(Convolutional Neural Network, CNN)的发展,现有的CNN已经能很好地抽取图片中的空间特征和语义特征,并将其用于特定的学习任务。因此,当前的基于表征学习的方法越来越多地将目光转向了基于CNN的特征表示。基于表征学习的方法所表示的特征根据其表达所涵盖的范围,又可分为全局特征和局部特征。全局特征指的是将整张图片看作一个整体,模型一次性地将图片的特征表示出来,最终生成的特征表示虽然也有可能会有图片局部的特征,但该局部的位置不是固定的,并不能稳定、有效地表达人体各个部位有别于其他行人的特征。与此相反,局部特征是将行人图像划分切块后,对于每个局部进行特征提取,划分的方式可以是简单的均匀划分\cite{sun2017beyond},也可以是不均匀的(例如根据人体姿势骨架关键点\cite{zheng2017pose})划分。在Sun等人\cite{sun2017beyond}的工作中,既实现了一个基于均匀分块的强有力的基线模型(baseline model),使用非常简单、直观的方法取得了非常稳定、出色的效果;又提出了一种将行人图片中不同的像素块分类到行人不同部分的方法,使得行人的特征表示在基线模型的基础上更加稳健,该模型是行人重识别领域中最新的成果(state-of-the-art)。

基于度量学习的方法致力于找到一种合理的度量方式,使得相同行人特征之间的距离尽可能近,不同行人特征之间的距离尽可能远。

\subsection{多摄像头选择}

\subsection{多CPU集群}

\section{论文解决的问题和挑战}
本节对论文要解决的问题进行形式化表述。

\section{论文的主要工作和创新点}
本节介绍了论文的主要工作。

\section{论文的组织结构}
论文的剩余部分的结构如下。