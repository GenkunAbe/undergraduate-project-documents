\chapter{绪论}
本章首先阐述了论文的研究背景,说明论文的动机和主要观点。接着介绍了与论文主题密切相关的前人工作,分析评价其贡献与不足。然后形式化地对要解决的问题进行定义,明确研究工作的界限和规模。最后介绍论文的主要工作,以及论文的组织结构。

\section{研究背景}

随着视频监控技术的发展,无人值守的视频监控设备被越来越普遍地部署在国民社会的各个方面,在公安、交通、智能楼宇、金融、司法、教文卫等领域都有着不可替代的作用。具体的应用场景包括平安城市、卡口系统、工地监控、自助银行、监狱劳教和学前教育等。

在视频监控领域一个很重要且极具挑战性的问题是行人的重识别。行人重识别,指的是在多个视野不重叠的监控视频中,重新识别那些之前出现过的行人,即把当前行人与之前已标记的人物相对应。该工作的实现可以为人物搜索、特定人物跟踪等应用提供强有力的支持,进而应用在平安城市、工地监控、学前教育等场景。

行人重识别技术在实际应用中受诸多因素的影响,包括摄像头的部署位置、成像质量以及摄像头数量等等。面对一个从未部署过摄像头的监控场景,一个优秀的摄像头位置部署方案可以实现监控范围无死角、行人重识别准确率高以及跨摄像头持续跟踪的连续性强。摄像头的成像质量受其自身的性能参数影响,同时也受部署位置的光线条件影响。在理想情况下,摄像头的数量越多,得到的行人信息就越多,更有利于行人重识别算法的实施。但是在预算有限的情况下,摄像头的数量不可能无限增加。因此,如何选择摄像头的数量及其部署位置,使得行人重识别算法的性能最大化,便成为一个极具现实意义的研究问题。

要实现行人的重识别需要借助计算机视觉领域的技术。目前的计算机视觉算法的前沿关注点主要集中在深度学习技术。由于深度学习的海量计算需求和GPU强大的并行计算能力,大部分深度学习模型都是借助GPU来调整模型参数。然而使用GPU进行运算也存在许多问题,例如GPU设备通常比较昂贵,且少见于嵌入式设备中。同时GPU的显存普遍不高,限制了模型的规模以及每一批次数据的规模。与此同时,多CPU集群具有硬件成本低、搭建方式灵活以及部署广泛等优点,然而对于深度学习算法在多CPU集群上的研究与应用少之又少。因此,有必要对深度学习算法在多CPU集群上的性能表现进行深入的调查、分析和研究。

\section{研究现状}
本节介绍了前人相关工作。

\subsection{行人重识别}

\subsection{多摄像头选择}

\subsection{多CPU集群}

\section{论文解决的问题和挑战}
本节对论文要解决的问题进行形式化表述。

\section{论文的主要工作和创新点}
本节介绍了论文的主要工作。

\section{论文的组织结构}
论文的剩余部分的结构如下。