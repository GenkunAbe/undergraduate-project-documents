\chapter{绪论}\label{sec:introduction}

本章首先阐述了论文的研究背景,说明论文的动机和主要观点。接着介绍了与论文主题密切相关的前人工作,分析评价其贡献与不足。然后形式化地对要解决的问题进行定义,明确研究工作的界限和规模。最后介绍论文的主要工作,以及论文的组织结构。

\section{研究背景}

随着视频监控技术的发展,无人值守的视频监控设备被越来越普遍地部署在国民社会的各个方面,在公安、交通、智能楼宇、金融、司法、教文卫等领域都有着不可替代的作用。具体的应用场景包括平安城市、卡口系统、工地监控、自助银行、监狱劳教和学前教育等。

在视频监控领域一个很重要且极具挑战性的问题是行人的重识别。行人重识别,指的是对多个视野不重叠的监控摄像头拍摄的同一个行人的图片进行匹配~\cite{chen2018person},即重新识别一些之前出现过的行人。该工作的实现可以为人物搜索、特定人物跟踪等应用提供强有力的支持,进而应用在平安城市、工地监控、学前教育等场景。

行人重识别技术在实际应用中受诸多因素的影响,包括摄像头的部署位置、成像质量以及摄像头数量等等。面对一个从未部署过摄像头的监控场景,一个优秀的摄像头位置部署方案可以实现监控范围无死角、行人重识别准确率高以及跨摄像头持续跟踪的连续性强的效果。影响摄像头成像质量的因素有很多,包括摄像头自身的性能参数,摄像头部署角度不同导致的不同视野范围,以及摄像头的部署位置不同导致的不同光线条件。在理想情况下,摄像头的数量越多,得到的行人信息就越多,更有利于行人重识别算法的实施。但是在预算有限的情况下,摄像头的数量不可能无限增加。因此,如何选择摄像头的数量及其部署位置,使得行人重识别算法的性能最大化,便成为一个极具现实意义的研究问题。

要实现行人的重识别需要借助计算机视觉领域的技术。目前的计算机视觉算法的前沿关注点主要集中在深度学习技术。由于深度学习的海量计算需求和GPU强大的并行计算能力,大部分深度学习模型都是借助GPU来调整模型参数。然而使用GPU进行运算也存在许多问题,例如GPU设备通常比较昂贵,且少见于嵌入式设备中。同时GPU的显存普遍不高,限制了模型的规模以及每一批次数据的规模。与此同时,多CPU集群具有硬件成本低、搭建方式灵活以及部署广泛等优点,然而对于深度学习算法在多CPU集群上的研究与应用少之又少。因此,有必要对深度学习算法在多CPU集群上的性能表现进行深入的调查、分析和研究。

\section{研究现状}
本节介绍了与本课题密切相关的三个领域,行人重识别、多摄像头选择以及多CPU集群,其中的前人工作以及研究现状。

\subsection{行人重识别}

行人重识别的任务是将多个视野不重叠的监控摄像头所拍摄的同一行人进行匹配~\cite{chen2018person}。与用于身份认证的人脸识别技术不同的是,行人重识别技术不是针对某个特定的行人进行建模,因为测试环节的数据可能没有在训练环节出现过。因此,行人重识别模型需要学习到一种判别两张图片空间特征和语义特征相似度的能力。要拥有该能力,则需要行人重识别模型具有强大的特征提取功能和距离度量功能。一个优秀的特征提取模型能够将行人图片中最具有判别性的特征很好地提取和表示,使得在所构造的特征空间内,不同行人所对应的特征的分界面尽可能地宽阔。一个优秀的距离度量模型能够使得相同行人的特征之间的距离尽可能近,不同行人的特征之间的距离尽可能远,即在特征空间内能够很好的将行人的特征进行正确地聚类。基于这样两个通用的建模阶段,在行人重识别领域研究的方法也主要分为两大类:基于表征学习(Representation Learning)的方法和基于度量学习(Metric Learning)的方法。

基于表征学习的方法将主要精力用于使模型拥有强有力的特征表示能力,使模型能够发现并表示行人图片中具有判别性的空间特征(如肤色、头发、衣着等)和语义特征(如性别、姿态、行为等)。得益于近年来深度学习~\cite{simonyan2014very},尤其是深度卷积神经网络~\cite{krizhevsky2012imagenet}(Convolutional Neural Network, CNN)的发展,现有的CNN已经能很好地抽取图片中的空间特征和语义特征,并将其用于特定的学习任务。因此,当前的基于表征学习的方法越来越多地将目光转向了基于CNN的特征表示。基于表征学习的方法所表示的特征根据其表达所涵盖的范围,又可分为全局特征和局部特征。全局特征指的是将整张图片看作一个整体,模型一次性地将图片的特征表示出来,最终生成的特征表示虽然也有可能会有图片局部的特征,但该局部的位置不是固定的,并不能稳定、有效地表达人体各个部位有别于其他行人的特征。与此相反,局部特征是将行人图像划分切块后,对于每个局部进行特征提取,划分的方式可以是简单的均匀划分~\cite{sun2017beyond},也可以是不均匀的(例如根据人体姿势骨架关键点~\cite{zheng2017pose})划分。在Sun等人~\cite{sun2017beyond}的工作中,既实现了一个基于均匀分块的强有力的基线模型(baseline model),使用非常简单、直观的方法取得了非常稳定、出色的效果;又提出了一种将行人图片中不同的像素块分类到行人不同部分的方法,使得行人的特征表示在基线模型的基础上更加稳健,该模型是行人重识别领域中的最新成果(state-of-the-art)。

基于度量学习的方法致力于找到一种合理的度量方式,使得相同行人特征之间的距离尽可能近,不同行人特征之间的距离尽可能远。基于度量学习的方法主要分成两大类:注重可解释性的传统的字典学习方法,以及基于深度学习的损失方法。字典学习方法将原始行人图片的特征向量映射到一个新的向量空间,在该向量空间中相同行人特征向量之间的距离普遍小于不同行人特征向量之间的距离。字典学习经常使用稀疏编码~\cite{lee2007efficient}的思想实现。字典学习的训练过程通常会配合一个带有明确目的性的损失函数来实现,例如在Li等人~\cite{li2018discriminative}的工作中通过半耦合的字典学习方法,设计了一个能够降低同一行人的不同分辨率图片之间的特征差异的损失函数,提升了行人重识别算法在不同分辨率下的识别效果。基于深度学习的损失方法主要将精力用于构造一个适合的损失函数,利用损失函数来引导深度神经网络的训练。常见的损失函数包括分类损失、三元组损失~\cite{schroff2015facenet}、四元组损失~\cite{chen2017beyond}以及边界挖掘损失~\cite{xiao2017margin},这些损失函数大多受人脸识别领域研究的启发,为特定的研究目标针对性地设计损失函数,借助深度神经网络强大的学习能力,取得了优异的效果。

从当前的研究形势来看,基于度量学习的方法同样依赖一个强大的表征学习模型,才能从原始行人图片中提取到合适的特征表达。同时目前基于度量学习的方法也越来越多地使用深度神经网络来作为前端,也可以把前端看作是一个特征提取的过程。因此,深度神经网络这一工具将会越来越多地被用于行人重识别领域,同时也会促进表征学习和度量学习的融合。然而基于深度学习的行人重识别方法也存在一些问题,例如其训练和预测过程的计算量过于庞大,且高度依赖GPU,不适合部署在实时视频监控场景。

\subsection{多摄像头选择}

近年来,在行人重识别领域出现了一些多摄像头的数据集,如Zheng等人公开的Market1501~\cite{zheng2015scalable}和Gou等人公开的DukeMTMC4ReID~\cite{gou2017dukemtmc4reid}数据集,它们分别包含6个和8个摄像头。然而这些数据集的评估协议(Evaluation Protocol)与双摄像头的行人重识别数据集的评估协议一脉相承,最大的区别仅仅在于多摄像头评估协议会将排序结果中来自同一摄像头的同一行人图像排除。这样的评估方式没有很好地利用多个摄像头的信息,且与多摄像头在实际监控场景中的需求不符。例如在实际的监控场景中,往往要求多个摄像头所拍摄的同一行人图片都识别准确,才算这一个行人的识别结果正确,这样的要求大大提高行人重识别问题的难度和挑战性,也更符合实际应用。要研究摄像头的数量与部署位置对行人重识别算法的影响,数据集需要满足以下要求:一、数据集中摄像头数量大于10,以便控制增减摄像头的进行分析对比,分析摄像头的拍摄位置对于监控效果的影响;二、存在视野重叠或拍摄位置接近的摄像头,以便分析同一位置、不同拍摄角度的拍摄方案对于监控效果的影响。三、行人需要出现在多个(而不仅仅是两个)摄像头画面内,且摄像头的分布呈路线状(可以存在分支),这样更符合行人跟踪的定义,更好地评估摄像头部署位置的监控效果。

多摄像头选择问题虽然在行人重识别领域尚欠研究,但是在目标追踪领域已有不少研究成果。在多摄像头目标追踪领域,多个摄像头会组成一个摄像头网络,各摄像头同步地拍摄视频画面,并与其它摄像头交换信息,以提升目标追踪的准确率。Bernab{\'e}等人~\cite{de2012entropy}提出了基于信息熵的算法,该算法动态地选择一些摄像头,在一定的传输带宽限制、传输错误率限制以及运算量限制下,达到最优的准确率。如果选择某个摄像头所带来的准确率收益大于该摄像头产生的传输和运算量负担,那么就会选择该摄像头。Bhuvana等人~\cite{bhuvana2016multi}提出了新颖度(surprisal)的概念,每个摄像头可以独立地计算其所拍摄的画面中的信息量,并且根据期望的摄像头数量,动态确定一个信息量阈值,只有当该摄像头画面中的信息量大于该阈值,才会选择该摄像头。在目标追踪领域的多摄像头选择研究也存在着一些不适用于行人重识别问题的情况和假设,例如在目标追踪问题中多个摄像头的视野都是重叠的,行人视频画面都是同步的,如此一来多个摄像头仅仅是同一行人、同一时刻的不同角度信息,不符合行人重识别领域中视野不重叠的要求。更进一步地,在同一时刻各摄像头只会拍到一个行人,所以摄像头之间不存在行人重识别的问题,只需要准确定位目标的位置。

行人重识别领域的多摄像头选择问题是一个非常具有现实意义的问题,但由于以往数据库中摄像头个数的限制,以及现有的评估方式的不足,所以该问题至今还没有被深入地研究。因此,有必要针对上述需求重新采集一个数据库,满足研究和实验的需求。

\subsection{多CPU集群}

自上个世纪以来,半导体技术和制造工艺飞速发展,芯片制造成本大大降低,当前的半导体技术和制造工艺也越来越接近经典物理定律的极限,因此人们把目光由单核处理器转向了多核处理器,以及多CPU集群计算。在解决CPU集群中各节点通信问题时,主要采用两种方案:MPI~\cite{sur2006high}和OpenMP~\cite{dagum1998openmp}。其中MPI是一个消息传递接口,OpenMP是共享内存编程的工业标准。基于这些接口和标准,出现了很多运行在多CPU集群上的应用~\cite{rabenseifner2009hybrid,ayguade2009design}。然而多CPU集群在深度学习方面的应用少之又少,其重要原因为GPU强大的并行计算能力。然而GPU计算也存在不足之处,如价格昂贵、配置复杂、显存不足等。这些问题正好是多CPU集群的优势。因此,分析、研究深度学习在多CPU集群上的性能表现是一件非常有意义的事情。

\section{研究问题}

本节对论文要解决的问题进行形式化表述。

给定源数据集$\mathcal{D}^{src}$,包含多标签、多摄像头、标注质量高的行人重识别数据集,用于行人重识别基线模型的训练。数据集$\mathcal{D}^{src}$需要包括训练集$\mathcal{D}_{train}^{src}$、测试查询集$\mathcal{D}_{query}^{src}$以及测试候选集$\mathcal{D}_{gallery}^{src}$。其中训练集$\mathcal{D}_{train}^{src}$为包含行人标签的图片集,方便用于训练阶段的标签分类任务。测试查询集$\mathcal{D}_{query}^{src}$包含多张标注有标签和摄像头的ground truth的图片,在模型评估阶段,将每张图片$\mathcal{D}_{query,\,(i)}^{src}$输入模型,模型从测试候选集$\mathcal{D}_{gallery}^{src}$中抽取出与$\mathcal{D}_{query,\,(i)}^{src}$相似的图片,并按照相似度进行排序,作为行人重识别任务的结果。

给定目标数据集$\mathcal{D}^{des}$,包含至少10个摄像头的行人重识别数据集,用于研究多摄像头选择问题。与源数据集中用于评估模型的部分类似,目标数据集$\mathcal{D}^{des}$也包含测试查询集$\mathcal{D}_{query}^{des}$以及测试候选集$\mathcal{D}_{gallery}^{des}$。考虑到更具有挑战性的行人追踪问题,目标数据集$\mathcal{D}^{des}$中的每张行人图片需要包含其摄像头信息,以及其在原视频中的帧序号信息。

评估指标$\mathcal{E}:\,\mathcal{F}(\mathcal{I})\mapsto\mathbb{R}$,其中$\mathcal{I}$是多个视野不重叠、时间连续的摄像头所拍摄的画面帧的集合,记录了某个行人从第一个摄像头走到最后一个摄像头的过程。$\mathcal{F}(\mathcal{I})$表示模型的预测结果。评估指标$\mathcal{E}$对于模型给出的预测结果,给出一个合理的、与摄像头选择目的相一致的评价。

本文的目标是在一定的约束下(如摄像头个数、成本、物理空间分布),探索摄像头部署方案。每一个摄像头部署方案将会对应一个目标数据集$\mathcal{D}^{des}$中的视频画面帧$\mathcal{I}$,使用经源数据集$\mathcal{D}^{src}$预训练的模型$\mathcal{F}$得到行人重识别结果,由评估指标$\mathcal{E}$对结果进行评价,找到评价最高的摄像头部署方案。

\section{论文的主要工作和创新点}

本节介绍了论文的主要工作和创新点。其中本文的主要工作如下:

首先,本文实现了行人重识别领域最新的研究成果~\cite{sun2017beyond},并在数据集~\cite{zheng2015scalable}达到了预期的结果,同时将算法应用在多CPU集群中,并与单机以及GPU进行比较。

其次,本文推出了一个全新的、包含多个摄像头的数据集的初始版本,该数据集包含16个摄像头,一共有5个场景,每个场景存在2$\sim$4个摄像头,适用于研究行人重识别领域中多摄像头选择问题。该数据集通过重新定义评估协议,规定所有摄像头的行人重识别结果同时正确才算正确,从而用于普通的行人重识别模型评估,增加行人重识别问题的挑战性和实用性。

最后,本文提出了基于强化学习的摄像头部署方案选择模型,面对一个未知的状态空间环境,模型中的智能体能够通过强化学习的方式认识到当前的状态的长期价值,并主动向长期价值更高的状态转移。该模型在上述数据集上取得了预期的结果,能够为摄像头部署方案的选择提供帮助,特别是在大规模摄像头部署过程中,能够有效降低运算量。

\section{论文的组织结构}

论文剩余部分的结构如下:第~~\ref{sec:theory}~~章讲述了本文主要工作的理论框架、模型和算法,第~~\ref{sec:algorithm}~~章详细讲述了项目中各工作的算法及其实现,以及实现过程中使用的协议和参数。第~~\ref{sec:experiment}~~章展示了详细的实验结果,并做了详细地分析和讨论。第~~\ref{sec:conclusion}~~章总结了项目和论文,以及介绍了今后的工作。