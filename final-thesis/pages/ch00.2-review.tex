\reviewschool{电子与信息工程}
\reviewmajor{计算机科学与技术}
\reviewclass{计算机44}
\reviewauthor{李源勋}
\reviewtitle{面向多CPU集群的深度行人重识别研究}
\reviewteachercomment{
    论文针对无人值守的视频监控,实现了行人重识别领域最新的研究成果,同时推出了一个全新的、包含多个摄像头的数据集的初始版本,提出了基于强化学习的摄像头部署方案选择模型。论文还展示了将模型部署在多CPU 集群上训练的结果,研究了多CPU 集群对于深度神经网络的训练过程的影响。论文工作难度较大,工作量饱满。\\
    该生在毕业设计工作中学习态度积极主动,工作认真努力。从论文完成情况看,该生掌握了扎实的专业基础理论和专业知识,具有较强的独立工作能力。论文达到了工学学士学位要求。专业英语文献翻译良好。
    }
\reviewteachergrade{优秀}
\reviewteachername{何晖}
\reviewteacheryear{2018}
\reviewteachermonth{6}
\reviewteacherday{11}

\reviewercomment{论文实现了行人重识别领域最新的研究成果,推出了一个全新的、包含多个摄像头的数据集的初始版本,提出了基于强化学习的摄像头部署方案选择模型,并将该模型部署在多CPU 集群上进行训练,研究了多CPU 集群对于深度神经网络的训练过程的影响。论文层次结构安排合理,表述清晰,文字流畅,达到了本科毕业设计(论文)要求 。}
\reviewergrade{优秀}
\reviewername{侯迪}
\reviewertitle{副教授}
\revieweryear{2018}
\reviewermonth{6}
\reviewerday{11}

\resultschool{电子与信息工程学}
\resultmajor{计算机科学与技术}
\resultauthor{李源勋}
\resulttitle{面向多CPU集群的深度行人重识别研究}
\resultcomment{}
\resultgrade{}
\resultprincipal{}
\resultmenberaname{}
\resultmenberatitle{}
\resultmenberbname{}
\resultmenberbtitle{}
\resultmenbercname{}
\resultmenberctitle{}
\resultyear{2018}
\resultmonth{6}
\resultday{11}