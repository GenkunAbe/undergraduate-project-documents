% 标题,中文
\ctitle{面向多CPU集群的深度行人重识别研究}
\covertitlefirst{面向多CPU集群的}
\covertitlesecond{深度行人重识别研究}
\cschool{电信}
\cactualinstitution{电信学院}
\cmajor{计算机科学与技术}
\cclass{计算机44}
\cyear{\the\year}
\cmonth{\the\month}

% 作者,中文
\cauthor{李源勋}
\stuid{2140505083}

% 学科,中文,本科生不需要
\csubject{}

% 导师姓名,中文
\csupervisor{何\quad 晖}

% 关键词,中文。用全角分号「;」分割
% 研究生的应首先从《汉语主题词表》中摘选
\ckeywords{行人重识别;深度学习;强化学习;多摄像头选择;多CPU集群}

% 提交日期,本科生不需要
\cproddate{\the\year 年\the\month 月}

% 论文类型,中文,本科生不需要
% 从理论研究、应用基础、应用研究、研究报告、软件开发、设计报告、案例分析、调研报告、其它中选择
\ctype{}

% 论文标题,英文
\etitle{Research for Multi-CPU Cluster Based Person Re-Identification}

% 作者姓名,英文
\eauthor{Yuanxun Li}

% 学科,英文,本科生不需要
\esubject{}

% 导师姓名,英文
\esupervisor{Hui He}

% 关键词,英文。用半角分号和一个半角空格「; 」分割
\ekeywords{Person Re-Identification; Deep Learning; Reinforcement Learning; Multi-Cameras Selection; Multi-CPUs Cluster}

% 学科门类,英文
% 从Philosophy(哲学)、Economics(经济学)、Law(法学)、Education(教育学)、Arts(文学)、
%   Science(理学)、Engineering Science(工学)、Medicine(医学)、Management Science(管理学)中选择
\ecate{Engineering Science}

% 提交日期,英文,本科生不需要
% 应当和 cproddate 保持一致
\eproddate{\monthname{\month}\ \the\year}

% 论文类型,英文,本科生不需要
% 从Theoretical Research(理论研究)、Application Fundamentals(应用基础)、Applied Research(应用研究)、
%   Research Report(研究报告)、Software Development(软件开发)、Design Report(设计报告)、
%   Case Study(案例分析)、Investigation Report(调研报告)、其它(Other)中选择
\etype{}

% 摘要,中文。段间空行
\cabstract{
    随着视频监控技术的发展,无人值守的视频监控设备被越来越普遍地部署在国民社会的各个方面。在视频监控领域一个很重要且极具挑战性的问题是行人的重识别。行人重识别,指的是在多个视野不重叠的监控视频中,重新识别那些之前出现过的行人,即把当前行人与之前已标记的人物相对应。行人重识别还有挑战性任务将多个摄像头内的多个行人精确匹配、跟踪。更进一步地,如果行人的重识别技术所基于的深度学习模型需要大量的计算资源,超级计算机是计算机中功能最强、运算速度最快、存储容量最大的一类计算机,一般由多CPU集群组成。多用于国家高科技领域和尖端技术研究,是一个国家科研实力的体现,它对国家安全,经济和社会发展具有举足轻重的意义。目前由于成本问题,基于超算的深度学习架构还不够流行,这既限制了科研理论的发展,又没有使超算发挥其应有的作用。

    因此,本项目将探索在多CPU集群上的深度行人重识别问题,致力于在项目过程中发现和解决问题,推动理论的发展和实际的应用。

    本文实现了行人重识别领域最新的研究成果,同时推出了一个全新的、包含多个摄像头的数据集的初始版本,并提出了基于强化学习的摄像头部署方案选择模型,该模型在自有数据集上取得了理想的结果,与人类的认知达成一致,具有很好的可解释性。本文还展示了将模型部署在多CPU集群上训练的结果,研究了多CPU集群对于深度神经网络的训练过程的影响。

    本文工作的特点主要有以下几个方面。本文实现的行人重识别基线模型取得了接近原论文的准确率和识别性能,为后期进一步研究行人重识别领域其它问题提供了强有力的支持。面向多CPU集群的深度行人重识别模型训练验证了CPU和GPU在进行海量单精度浮点数运算的性能差异,同时也展示了多CPU集群在训练深度神经网络方面的巨大潜力。本文推出的行人重识别数据集包含了17个摄像头,并且还包含了完整的行人跟踪场景,不仅在摄像头数量上远远多于现有主流的行人重识别数据集,而且为行人重识别问题新的评估方式的提出提供了基础。本文提出的基于强化学习模型的摄像头部署方案选择模型,将强化学习算法运用到多摄像头选择问题当中,避免了穷举计算,且达到了理想的效果。
}

% 摘要,英文。段间空行
\eabstract{
    With the development of video surveillance technology, unattended video surveillance devices are being deployed more and more widely in various aspects of the civil society, and are irreplaceable in fields such as public security, transportation, intelligent buildings, finance, justice, and education. The role. Specific application scenarios include Pingan City, bayonet system, site monitoring, self-service banking, prison labor camps and pre-school education.

    A very important and challenging issue in video surveillance is pedestrian recognition. Pedestrian re-identification refers to the re-identification of pedestrians who have previously appeared in multiple surveillance videos that do not overlap, that is, to associate the current pedestrian with the previously marked person. The implementation of this work can provide strong support for applications such as character search and specific person tracking, and it can be used in scenes such as safe city, site monitoring, and preschool education.

    To achieve pedestrian re-identification requires the use of computer vision technology. The frontier focus of current computer vision algorithms is mainly on deep learning techniques. Due to the massive computing requirements of deep learning and the GPU's powerful parallel computing capabilities, most of the deep learning algorithms use GPUs to adjust model parameters. However, the current GPU has very limited RAM, which limits the depth and training speed of deep neural networks, and limits its research space and specific applications for many specific issues, including pedestrian re-identification issues.

    The supercomputer is a type of computer with the most powerful functions, the fastest computing speed, and the largest storage capacity. It is mostly used in the research of high-tech fields and cutting-edge technologies of the country. It is a manifestation of the strength of scientific research in a country. It is a national security, economic and social development. Has a significant significance. At present, because of the cost problem, the deep learning framework based on super-calculation is not yet popular enough. This not only limits the development of scientific research theory, but also does not make super-calculation play its due role.

    Therefore, this project will explore in-depth pedestrian re-identification issues on multi-CPU clusters, commit to discovering and resolving problems during the project process, and promote theoretical development and practical application.
}
