% meta.tex
%
% Aetf <aetf@unlimitedcodeworks.xyz>
% Copyright 2016 Aetf <aetf@unlimitedcodeworks.xyz>
%
% multiple1902 <multiple1902@gmail.com>
% Copyright 2011~2012, multiple1902 (Weisi Dai)
%
% Project Home: https://github.com/Aetf/xjtuthesis
%
% It is strongly recommended that you read documentations located at
%   https://github.com/Aetf/xjtuthesis/wiki
% in advance of your compilation if you have not read them before.
%
% This work may be distributed and/or modified under the
% conditions of the LaTeX Project Public License, either version 1.3
% of this license or (at your option) any later version.
% The latest version of this license is in
%   http://www.latex-project.org/lppl.txt
% and version 1.3 or later is part of all distributions of LaTeX
% version 2005/12/01 or later.
%
% This work has the LPPL maintenance status `maintained'.
%
% The Current Maintainer of this work is Aetf.
%

% 标题,中文
\ctitle{面向多CPU集群的深度行人重识别研究}
\covertitlefirst{面向多CPU集群的}
\covertitlesecond{深度行人重识别研究}
\cschool{电信}
\cactualinstitution{电信学院}
\cmajor{计算机科学与技术}
\cclass{计算机44}
\cyear{\the\year}
\cmonth{\the\month}

% 作者,中文
\cauthor{李源勋}
\stuid{2140505083}

% 学科,中文,本科生不需要
\csubject{}

% 导师姓名,中文
\csupervisor{何\quad 晖}

% 关键词,中文。用全角分号「;」分割
% 研究生的应首先从《汉语主题词表》中摘选
\ckeywords{行人重识别;深度学习;强化学习;多CPU集群}

% 提交日期,本科生不需要
\cproddate{\the\year 年\the\month 月}

% 论文类型,中文,本科生不需要
% 从理论研究、应用基础、应用研究、研究报告、软件开发、设计报告、案例分析、调研报告、其它中选择
\ctype{}

% 论文标题,英文
\etitle{Research for Multi-CPU Cluster Based Person Re-Identification}

% 作者姓名,英文
\eauthor{Yuanxun Li}

% 学科,英文,本科生不需要
\esubject{}

% 导师姓名,英文
\esupervisor{Hui He}

% 关键词,英文。用半角分号和一个半角空格「; 」分割
\ekeywords{Person Re-Identification; Deep Learning; Reinforcement Learning; Multi-CPU Cluster}

% 学科门类,英文
% 从Philosophy(哲学)、Economics(经济学)、Law(法学)、Education(教育学)、Arts(文学)、
%   Science(理学)、Engineering Science(工学)、Medicine(医学)、Management Science(管理学)中选择
\ecate{Engineering Science}

% 提交日期,英文,本科生不需要
% 应当和 cproddate 保持一致
\eproddate{\monthname{\month}\ \the\year}

% 论文类型,英文,本科生不需要
% 从Theoretical Research(理论研究)、Application Fundamentals(应用基础)、Applied Research(应用研究)、
%   Research Report(研究报告)、Software Development(软件开发)、Design Report(设计报告)、
%   Case Study(案例分析)、Investigation Report(调研报告)、其它(Other)中选择
\etype{}

% 摘要,中文。段间空行
\cabstract{
    随着视频监控技术的发展,无人值守的视频监控设备被越来越普遍地部署在国民社会的各个方面,在公安、交通、智能楼宇、金融、司法、教文卫等领域都有着不可替代的作用。具体的应用场景包括平安城市、卡口系统、工地监控、自助银行、监狱劳教和学前教育等。

    在视频监控领域一个很重要且极具挑战性的问题是行人的重识别。行人重识别,指的是在多个视野不重叠的监控视频中,重新识别那些之前出现过的行人,即把当前行人与之前已标记的人物相对应。该工作的实现可以为人物搜索、特定人物跟踪等应用提供强有力的支持,进而应用在平安城市、工地监控、学前教育等场景。

    要实现行人的重识别需要借助计算机视觉领域的技术。目前的计算机视觉算法的前沿关注点主要集中在深度学习技术。由于深度学习的海量计算需求和GPU强大的并行计算能力,大部分深度学习算法都是借助GPU来调整模型参数。然而目前的GPU的RAM十分有限,这就约束了深度神经网络的深度和训练速度,限制了其对于很多具体问题,包括行人重识别问题,的研究空间和具体应用。

    超级计算机是计算机中功能最强、运算速度最快、存储容量最大的一类计算机,多用于国家高科技领域和尖端技术研究,是一个国家科研实力的体现,它对国家安全,经济和社会发展具有举足轻重的意义。目前由于成本问题,基于超算的深度学习架构还不够流行,这既限制了科研理论的发展,又没有使超算发挥其应有的作用。

    因此,本项目将探索在多CPU集群上的深度行人重识别问题,致力于在项目过程中发现和解决问题,推动理论的发展和实际的应用。
}

% 摘要,英文。段间空行
\eabstract{
    With the development of video surveillance technology, unattended video surveillance devices are being deployed more and more widely in various aspects of the civil society, and are irreplaceable in fields such as public security, transportation, intelligent buildings, finance, justice, and education. The role. Specific application scenarios include Pingan City, bayonet system, site monitoring, self-service banking, prison labor camps and pre-school education.

    A very important and challenging issue in video surveillance is pedestrian recognition. Pedestrian re-identification refers to the re-identification of pedestrians who have previously appeared in multiple surveillance videos that do not overlap, that is, to associate the current pedestrian with the previously marked person. The implementation of this work can provide strong support for applications such as character search and specific person tracking, and it can be used in scenes such as safe city, site monitoring, and preschool education.

    To achieve pedestrian re-identification requires the use of computer vision technology. The frontier focus of current computer vision algorithms is mainly on deep learning techniques. Due to the massive computing requirements of deep learning and the GPU's powerful parallel computing capabilities, most of the deep learning algorithms use GPUs to adjust model parameters. However, the current GPU has very limited RAM, which limits the depth and training speed of deep neural networks, and limits its research space and specific applications for many specific issues, including pedestrian re-identification issues.

    The supercomputer is a type of computer with the most powerful functions, the fastest computing speed, and the largest storage capacity. It is mostly used in the research of high-tech fields and cutting-edge technologies of the country. It is a manifestation of the strength of scientific research in a country. It is a national security, economic and social development. Has a significant significance. At present, because of the cost problem, the deep learning framework based on super-calculation is not yet popular enough. This not only limits the development of scientific research theory, but also does not make super-calculation play its due role.

    Therefore, this project will explore in-depth pedestrian re-identification issues on multi-CPU clusters, commit to discovering and resolving problems during the project process, and promote theoretical development and practical application.
}
