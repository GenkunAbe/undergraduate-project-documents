\chapter{结论与展望}\label{sec:conclusion}

本文实现了行人重识别领域最新的研究成果,同时推出了一个全新的、包含多个摄像头的数据集的初始版本,并提出了基于强化学习的摄像头部署方案选择模型,该模型在自有数据集上取得了理想的结果,与人类的认知达成一致,具有很好的可解释性。本文还展示了将模型部署在多CPU集群上训练的结果,研究了多CPU集群对于深度神经网络的训练过程的影响。

本文工作的特点主要有以下几个方面。本文实现的行人重识别基线模型取得了接近原论文的准确率和识别性能,为后期进一步研究行人重识别领域其它问题提供了强有力的支持。面向多CPU集群的深度行人重识别模型训练验证了CPU和GPU在进行海量单精度浮点数运算的性能差异,同时也展示了多CPU集群在训练深度神经网络方面的巨大潜力。本文推出的行人重识别数据集包含了17个摄像头,并且还包含了完整的行人跟踪场景,不仅在摄像头数量上远远多于现有主流的行人重识别数据集,而且为行人重识别问题新的评估方式的提出提供了基础。本文提出的基于强化学习模型的摄像头部署方案选择模型,将强化学习算法运用到多摄像头选择问题当中,避免了穷举计算,且达到了理想的效果。

此项研究在未来还有多个可以扩展的方向。首先,行人重识别基线模型的模型结构以及超参数设置还有很大的调整空间,花费更多精力于此可以得到准确率更高的模型。其次,完整版本的多摄像头行人重识别数据库还有很多的数据预处理工作,以及在数据处理完成之后,在该数据集上还有更多可以研究的问题。还有,面向强化学习的摄像头部署方案选择模型,还可以往更深的网络结构探索,如何结合深度学习和强化学习的优势以应用于该特定问题,值得更加深入的研究。