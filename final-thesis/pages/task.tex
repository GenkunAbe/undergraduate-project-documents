\taskmajor{计算机科学与技术}
\taskpresident{唐亚哲}
\taskapprovaldate{2018-03-05}
\taskschool{电子与信息工程}
\taskclass{计算机~44}
\taskauthor{李源勋}
\tasktitle{面向多CPU集群的深度行人重识别研究}
\taskfromyear{2018}
\taskfrommonth{2}
\taskfromday{26}
\tasktoyear{2018}
\tasktomonth{6}
\tasktoday{1}
\taskplace{西安交通大学}

\taskbackground{
    \uline{随着视频监控技术的发展,无人值守的视频监控设备被越来越普遍地部署在国民社会的各个方面,在公安、交通、智能楼宇、金融、司法、教文卫等领域都有着不可替代的作用。具体的应用场景包括平安城市、卡口系统、工地监控、自助银行、监狱劳教和学前教育等。}

    \uline{在视频监控领域一个很重要且极具挑战性的问题是行人的重识别。行人重识别,指的是在多个视野不重叠的监控视频中,重新识别那些之前出现过的行人,即把当前行人与之前已标记的人物相对应。在此基础上,对于现实场景中监控设备的部署位置,也是一个非常值得研究的问题,该工作的实现可以为监控设备的部署位置提供参考,为人物搜索、特定人物跟踪等应用提供更加准确、高效的性能,进而应用在平安城市、工地监控、学前教育等场景。}

    \uline{要实现行人的重识别需要借助计算机视觉领域的技术。目前的计算机视觉算法的前沿关注点主要集中在深度学习技术。由于深度学习的海量计算需求和GPU强大的并行计算能力,大部分深度学习算法都是借助GPU来调整模型参数。然而目前的GPU的RAM十分有限,这就约束了深度神经网络的深度和训练速度,限制了其对于很多具体问题,包括行人重识别问题,的研究空间和具体应用。}

    \uline{超级计算机是计算机中功能最强、运算速度最快、存储容量最大的一类计算机,多用于国家高科技领域和尖端技术研究,是一个国家科研实力的体现,它对国家安全,经济和社会发展具有举足轻重的意义。目前由于成本问题,基于超算的深度学习架构还不够流行,这既限制了科研理论的发展,又没有使超算发挥其应有的作用。}

    \uline{因此,本项目将探索在多CPU集群上的深度行人重识别问题,致力于在项目过程中发现和解决问题,推动理论的发展和实际的应用。}
}

\taskrawmaterial{
    \uline{在行人重识别领域公认的用于评定一个模型效果的数据集有:VIPeR~\mbox{\cite{gray2007evaluating}}、CUHK01~\mbox{\cite{li2012human}}、CUHK03~\mbox{\cite{li2012human}}、Market-1501~\mbox{\cite{zheng2015scalable}}、DukeMTMC-reID~\mbox{\cite{ristani2016MTMC}}。}

    \uline{毕业设计的资料包括国际上前沿的计算机视觉领域,特别是有关行人重识别问题的期刊、会议论文。~\mbox{\cite{ren2015faster,li2014deepreid,ristani2016MTMC,sun2017beyond,he2017mask,he2016deep,chen2018person}}}
}

\taskmaintask{
    \begin{enumerate}
        \item \uline{将目前的深度行人重识别工程改造为适用于多CPU集群的形式。}
        \item \uline{将工程部署在天河二号集群中,解决过程中遇到的具体问题。}
        \item \uline{采集同一地点不同视角的摄像头监控录像,作为实验数据。}
        \item \uline{研究摄像头的部署位置对于行人重识别算法性能的影响,求解最优部署方案。}
        \item \uline{对工程进行速度和正确率方面的性能优化。}
        \item \uline{总结经验和结论。}
    \end{enumerate}
}

\taskrequirement{
    \begin{enumerate}
        \item \uline{能够将现有的行人重识别算法运行在多CPU集群上。}
        \item \uline{采集的监控录像符合实验要求。}
        \item \uline{对于摄像头部署位置的研究能够得出合理的实验结果。}
        \item \uline{对于实验结果得出合理的经验和结论。}
    \end{enumerate}
}

\taskfinalmaterial{
    \uline{毕业设计论文不少于 15000 字,英文文献翻译不少于 5000 字。}
}

\taskothertaskreference{
}