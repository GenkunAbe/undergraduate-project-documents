\topicschool{电子与信息工程学院}
\topicmajor{计算机科学与技术}
\topicteachername{何晖}
\topicteachertitle{助理研究员}
\topictitle{面向多CPU集群的深度行人重识别研究 }
\topicprojecttype{综合实验}
\topictopicsource{校外协作项目}
\topicstudentnum{1}

\topicpurpose{
    \qquad 随着视频监控技术的发展,无人值守的视频监控设备被越来越普遍地部署在国民社会的各个方面,在公安、交通、智能楼宇、金融、司法、教文卫等领域都有着不可替代的作用。具体的应用场景包括平安城市、卡口系统、工地监控、自助银行、监狱劳教和学前教育等。

    \qquad 在视频监控领域一个很重要且极具挑战性的问题是行人的重识别。行人重识别,指的是在多个视野不重叠的监控视频中,重新识别那些之前出现过的行人,即把当前行人与之前已标记的人物相对应。该工作的实现可以为人物搜索、特定人物跟踪等应用提供强有力的支持,进而应用在平安城市、工地监控、学前教育等场景。

    \qquad 要实现行人的重识别需要借助计算机视觉领域的技术。然而目前的GPU的RAM十分有限,这就约束了深度神经网络的深度和训练速度,限制了其对于很多具体问题,包括行人重识别问题,的研究空间和具体应用。

    \qquad 超级计算机是计算机中功能最强、运算速度最快、存储容量最大的一类计算机,多用于国家高科技领域和尖端技术研究,是一个国家科研实力的体现,它对国家安全,经济和社会发展具有举足轻重的意义。目前由于成本问题,基于超算的深度学习架构还不够流行,这既限制了科研理论的发展,又没有使超算发挥其应有的作用。

    \qquad 因此,本项目将探索在多CPU集群上的深度行人重识别问题,致力于在项目过程中发现和解决问题,推动理论的发展和实际的应用。
}

\topiccontent{
    1、将目前的深度行人重识别工程改造为适用于多CPU集群的形式。

    2、将工程部署在天河二号集群中,解决过程中遇到的具体问题。

    3、对工程进行速度和正确率方面的性能优化。

    4、总结经验并发展新的理论。
}

\topicbackground{
    1、解决深度行人重识别问题一般分为两个阶段:一是图像特征的表示,二是特征距离的衡量。在图像特征表示阶段目前主要的方法是使用深度卷积神经网络来对原始图像的轮廓、纹理、形状等特征进行抽取。在特征距离衡量阶段可以使用特征扩展、图像对齐等方法。

    2、目前基于多CPU的深度学习框架包括BigML和Intel-Caffe,这些框架主要致力于通用性和可靠性,所以很适合在这些框架的基础上进行进一步的开发。
}

\topicteacheryear{2017}
\topicteachermonth{12}
\topicteacherday{2}

\topicmajoropinion{同意}
\topicmajorpresident{唐亚哲}
\topicmajorpresidentyear{2017}
\topicmajorpresidentmonth{12}
\topicmajorpresidentday{4}

\topicschoolopinion{同意}
\topicschoolpresident{罗新民}
\topicschoolpresidentyear{2017}
\topicschoolpresidentmonth{12}
\topicschoolpresidentday{4}
