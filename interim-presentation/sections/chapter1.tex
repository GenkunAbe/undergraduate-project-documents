\section{项目背景及意义}

\subsection{行人重识别}

    \begin{frame}{行人重识别的研究内容}
    \begin{block}{}
    行人重识别,指的是在多个视野不重叠的监控视频中,重新识别那些之前出现过的行人,即把当前行人与之前已标记的人物相对应。
    \end{block}
    \end{frame}

    \begin{frame}{多集群CPU}
    \begin{block}{}
    要实现行人的重识别需要借助计算机视觉领域的技术。目前的计算机视觉算法的前沿关注点主要集中在深度学习技术。由于深度学习的海量计算需求和GPU强大的并行计算能力,大部分深度学习模型都是借助GPU来调整模型参数。然而使用GPU进行运算也存在许多问题,例如GPU设备通常比较昂贵,且少见于嵌入式设备中。同时GPU的显存普遍不高,限制了模型的规模以及每一批次数据的规模。与此同时,多CPU集群具有硬件成本低、搭建方式灵活以及部署广泛等优点,然而对于深度学习算法在多CPU集群上的研究与应用少之又少。因此,有必要对深度学习算法在多CPU集群上的性能表现进行深入的调查、分析和研究。
    \end{block}
    \end{frame}

\subsection{摄像头选择}

    \begin{frame}{行人重识别的研究内容}
    \begin{block}{}
    行人重识别,指的是在多个视野不重叠的监控视频中,重新识别那些之前出现过的行人,即把当前行人与之前已标记的人物相对应。
    \end{block}
    \end{frame}

\subsection{多集群CPU}

    \begin{frame}{行人重识别的研究内容}
    \begin{block}{}
    行人重识别,指的是在多个视野不重叠的监控视频中,重新识别那些之前出现过的行人,即把当前行人与之前已标记的人物相对应。
    \end{block}
    \end{frame}
