\section{项目背景及意义}

    \begin{frame}{行人重识别的研究内容}
    \begin{block}{}
    行人重识别,指的是在多个视野不重叠的监控视频中,重新识别那些之前出现过的行人,即把当前行人与之前已标记的人物相对应。
    \end{block}
    \begin{figure}
    \centering
    \includegraphics[width=0.6\textwidth,trim={0 200 0 0},clip=true]{figures/vis3}
    \caption{行人重识别解决的问题}
    \label{fig:reid_intro}
    \end{figure}
    \end{frame}


    \begin{frame}{摄像头选择}
    \begin{block}{}
    从下图可以看出,相同拍摄地点的不同摄像头所拍摄的画面之间存在较大差异,如角度、光线、拍摄范围与通过时长。这些差异可能会影响行人跟踪的效果。
    \end{block}
    \begin{table}
    \centering
    \begin{tabular}{c}
    \includegraphics[width=12mm]{figures/1-1}~\includegraphics[width=12mm]{figures/1-2} \\
    \includegraphics[width=12mm]{figures/1-4}~\includegraphics[width=12mm]{figures/1-5}~\includegraphics[width=12mm]{figures/1-6} \\
    \includegraphics[width=12mm]{figures/2-1}~\includegraphics[width=12mm]{figures/2-2}~\includegraphics[width=12mm]{figures/2-3}~\includegraphics[width=12mm]{figures/2-4} \\
    \includegraphics[width=12mm]{figures/3-1}~\includegraphics[width=12mm]{figures/3-2}~\includegraphics[width=12mm]{figures/3-3} \\
    \includegraphics[width=12mm]{figures/3-4}~\includegraphics[width=12mm]{figures/3-5}~\includegraphics[width=12mm]{figures/3-6}~\includegraphics[width=12mm]{figures/3-7} \\
    \end{tabular}
    \end{table}
    \end{frame}


    \begin{frame}{多集群CPU}
    \begin{block}{}
    目前大部分深度学习模型都是借助GPU来调整模型参数。\\
    GPU 缺点:
    \begin{enumerate}
        \item GPU设备通常比较昂贵,且少见于嵌入式设备中
        \item GPU的显存普遍不高,限制了模型的规模及数据量
    \end{enumerate}
    多CPU集群优势:
    \begin{enumerate}
        \item 硬件成本低、搭建方式灵活以及部署广泛
        \item 对于深度学习算法在多CPU集群上的研究与应用少之又少
    \end{enumerate}
    \end{block}
    \end{frame}
