\pagenumbering{arabic}
\section{项目背景及内容}
随着视频监控技术的发展,无人值守的视频监控设备被越来越普遍地部署在国民社会的各个方面,在
公安、交通、智能楼宇、金融、司法、教文卫等领域都有着不可替代的作用。具体的应用场景包括平
安城市、卡口系统、工地监控、自助银行、监狱劳教和学前教育等。

在视频监控领域一个很重要且极具挑战性的问题是行人的重识别。行人重识别,指的是在多个视野不
重叠的监控视频中,重新识别那些之前出现过的行人,即把当前行人与之前已标记的人物相对应。该
工作的实现可以为人物搜索、特定人物跟踪等应用提供强有力的支持,进而应用在平安城市、工地监
控、学前教育等场景。

行人重识别技术在实际应用中受诸多因素的影响,包括摄像头的部署位置、成像质量以及摄像头数量
等等。面对一个从未部署过摄像头的监控场景,一个优秀的摄像头位置部署方案可以实现监控范围无
死角、行人重识别准确率高以及跨摄像头持续跟踪的连续性强。摄像头的成像质量受其自身的性能参
数影响,同时也受部署位置的光线条件影响。在理想情况下,摄像头的数量越多,得到的行人信息就
越多,更有利于行人重识别算法的实施。但是在预算有限的情况下,摄像头的数量不可能无限增加。
因此,如何选择摄像头的数量及其部署位置,使得行人重识别算法的性能最大化,便成为一个极具现
实意义的研究问题。

要实现行人的重识别需要借助计算机视觉领域的技术。目前的计算机视觉算法的前沿关注点主要集中
在深度学习技术。由于深度学习的海量计算需求和GPU强大的并行计算能力,大部分深度学习模型都
是借助GPU来调整模型参数。然而使用GPU进行运算也存在许多问题,例如GPU设备通常比较昂贵,
且少见于嵌入式设备中。同时GPU的显存普遍不高,限制了模型的规模以及每一批次数据的规模。与
此同时,多CPU集群具有硬件成本低、搭建方式灵活以及部署广泛等优点,然而对于深度学习算法在
多CPU集群上的研究与应用少之又少。因此,有必要对深度学习算法在多CPU集群上的性能表现进行
深入的调查、分析和研究。

基于上述研究背景及动机,本项目的主要研究内容和贡献如下:

\begin{enumerate}
\item 深入理解论文\cite{sun2017beyond}的目的、想法和实现方式,并使用PyTorch深度学习框架
复现,达到与原论文接近的实验结果。
\item 研究在预算有限的情况下,摄像头的数量与部署位置对行人重识别算法的影响,并采用强化学习
算法进行摄像头部署位置的选择。
\item 调查、分析和研究深度行人重识别算法在多CPU集群上的性能表现,提出针对多CPU集群环境
的深度神经网络模型的改进策略。
\end{enumerate}

本项目当前的进展:

\begin{enumerate}
\item 搭建了PyTorch分布式环境,完成了论文复现工作,在Market1501\cite{zheng2015scalable}
数据库上进行实验。
\item 实现了 Q-Learning 强化学习算法,选择最优的摄像头部署方案。
\item 将深度行人重识别模型的训练算法改造成适用于多CPU集群的形式。
\end{enumerate}

























