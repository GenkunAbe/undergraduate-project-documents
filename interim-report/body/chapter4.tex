\section{项目内容}

\subsection{数据采集}

\subsection{数据预处理}

\subsection{行人重识别领域state-of-the-art论文复现}

\subsubsection{模型训练}

\subsubsection{特征提取}

\subsection{强化学习框架实现}

\subsection{面向CPU集群的分布式深度学习训练框架}


\subsection{代码编写}
有了对论文思想的理解,再加上PyTorch这一强大的工具,代码实现起来就没那么困难了。
PyTorch采用面向对象的思想,其中的模型和数据集都是以类的形式存在的,所以需要
先写一个网络模型类和一个数据加载类。

\subsubsection{改造神经网络模型}
不同于 Caffe 把网络模型参数定义在一个配置文件内,以及 TensorFlow 的面向过程的搭建方式,
PyTorch 里面网络模型的定义采用面向对象的方法,只需要继承 torch.nn.Module 类,并且重载
里面的 \_\_init\_\_\(\) 方法和 forward\(\) 方法,便完成了一个神经网络的构造。\\

\begin{python}
class Net(nn.Module):
    def __init__(self, out_size=1501):
        super(Net, self).__init__()
        resnet = resnet50(pretrained=True)
        backbone_model = nn.Sequential(*list(resnet.children())[:-2])
        self.resnet = backbone_model
        self.resnet = self.resnet
        init_val = [torch.zeros(24, 8) for _ in range(6)]
        self.Ws = nn.ParameterList([Parameter(init_val[i]) for i in range(6)])
        self.avgpool = nn.AvgPool2d((24, 8))
        self.conv1 = nn.Conv2d(2048, 256, 1)
        self.fcs = nn.ModuleList([nn.Linear(256, out_size) for _ in range(6)])

    def forward(self, x):
        x = self.resnet.forward(x)
        xs = [None for _ in range(6)]
        for i in range(6):
            x_i = torch.mul(x, self.Ws[i])
            x_i = self.avgpool(x_i)
            x_i = F.relu(self.conv1(x_i))
            x_i = x_i.view(-1, 256)
            x_i = self.fcs[i](x_i)
            xs[i] = x_i
        return xs
\end{python}

\subsubsection{读入训练/测试数据}
在行人重识别领域比较有影响力的数据集有Market1501,数据集的格式是原始的JPG图片,所以
需要加载自己的训练集,这种情况下最好还是继承 Dataset 类比较方便。

Dataset 类的本质是定义了数据所在的位置,以及数据需要预处理的方法。至于数据的位置是在
硬盘里面还是提前加载到内存里面,由该类的内部实现决定。

\subsubsection{训练过程代码}

普通的单机训练过程主要定义损失函数、优化器,然后将输入数据经过网络模型的forward过程
得到输出,用损失函数计算误差,将误差通过backward算得每一个中间参数的梯度,最后使用
梯度下降优化器更新参数的梯度,完成一次迭代训练。

若是使用GPU训练,则需要将模型和数据都放入GPU,在GPU内进行forward和backward操作。

若是使用分布式训练,则需要初始化各节点与master节点的通信连接,然后等待master节点发来
的训练任务。

\subsubsection{测试过程代码}

测试采用了与原论文中相同的Rank1、Rank5、Rank10和mAP指标,保证复现结果的正确性。
