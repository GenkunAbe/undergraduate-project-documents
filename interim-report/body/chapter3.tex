\section{理论框架}
项目的理论框架主要包括当前行人重识别领域state-of-the-art的算法思想、摄像头部署方案的评价指标以及强化学习模型在项目中的应用。

\subsection{行人重识别领域state-of-the-art的算法思想}

\subsubsection{Part-based Convolutional Baseline (PCB)}
这篇论文的Baseline采用了最近很热门的Part的思想,但不同于计算图片中的Attention,而是简单地将图片在垂直方向上分块,获取每一个块的特征。

图\ref{fig:baseline} 是Baseline的架构图。在Baseline中,模型以ResNet50作为Backbone Network。ResNet50模型首先在ImageNet数据集上训练至收敛,然后去掉为ImageNet分类任务而设计的全局池化层(Global Average Pooling Layer)及其后面的全连接层(Fully Connected Layer),使ResNet50模型成为一个高效的图像特征提取器,其中的特征既包括颜色、纹理、形状等视觉特征,也包括类别、姿势、性别等语义特征。作为一个端到端(End-to-End)的特征提取器,其输入为包含RGB通道的原始图像,输出为包含2048个通道(2048 Channels)的Feature Maps。

将ResNet50模型输出的Feature Maps在竖直方向上分成$p=6$个水平条(Horizontal Stripes),每个通道(Channel)保持独立。每个水平条通过一个尺寸与水平条尺寸相同全局池化层,使得原本为矩形的水平条变为一个$1\times1$的像素点,再将其与同一水平条其他通道的像素点拼接起来得到一个$2048\times1\times1$的向量。因为每个水平条会得到一个特征向量,所以经过全局池化层之后可得到$p$个向量,每一个向量都能表示原图像在对应的水平条范围内的局部特征。此方法的优点是简单、高效、易实现,缺点是每个人各部位的分布不同,人物上所具备的关注点也千差万别,将Feature Maps在竖直方向上均匀分割不能很好地体现人与人之间的这些差异。

得到$p$个2048维的特征向量之后,再使用核尺寸为$1\times1$的卷积层将每个2048维的向量降为256维,以减少之后分类任务的计算量。每一个局部特征向量后接一个$n$分类器以预测该图像的类别,其中$n$为训练集中label的个数。训练的损失函数(Loss Function)使用交叉熵损失(Cross Entropy Loss)。

在测试阶段,也即特征提取阶段,将最后的$p$个$n$分类器去掉,直接将$p$个256维的向量拼接(concatenate)为向量$\textbf g$或将$p$个2048维的向量拼接为向量$\textbf h$作为原始行人图像的特征表示。

\subsubsection{Refined Part Pooling (RPP)}
Part-based Convolutional Baseline (PCB) 将Feature Maps在竖直方向上均匀分成$p$个水平条,以获得行人各部位的特征。此方法有操作简单、运算量少、易于实现的优点,但忽略了人与人之间各部位的位置差距。因此,有必要在Baseline的基础上进行改进,不再局限于一个规范的矩形,而是通过计算判断每一个像素点「应该属于」哪一个部分。如图\ref{fig:refined}所示。对于不同通道、相同位置的像素点,进行统一处理。

于是目标就成了:给定一个列向量(column vector,表示不同通道、相同位置的像素点),判断其属于哪一个部分。这就变成了一个分类问题。在这里使用一个线性神经网络层,来将所有的列向量分类。线性神经网络层有权重$W$和偏置$b$组成,为了简化表示,这里省略偏置$b$。对于一个列向量$f$,其属于第$i$个部分的概率$P(p_i\,\vert\,f)$为:
\begin{equation}
P(p_i\,\vert\,f)=\mathop{\rm softmax}\left(W_i^{\rm T}f\right)=\frac{\exp\left(W_i^{\rm T}f\right)}{\sum_j^p\exp\left(W_j^{\rm T}f\right)}
\end{equation}
其中$p_i$表示Feature Map的第$i$部分。

在PCB中,列向量$f$只绝对的属于某一个部分。在求得$P(p_i\,\vert\,f)$后,则可将原始的列向量$f$按照概率分布分配到各部分。对于第$i$个部分$p_i$,其计算方式为:
\begin{equation}
p_i=\frac{1}{H\times W}\sum_{j=1}^{H\times W}P(p_i\,\vert\,f_j)\times f_j
\end{equation}
其中$H$、$W$分别代表Feature Map的高和宽。

如此一来即可得到与PCB算法等价的输出,再经过与PCB算法后半部分相同的$1\times1$卷积层和分类器,便完成了Refined Part Pooling (RPP)算法的训练模型。完整的PCB+RPP架构图如图\ref{fig:structure2}所示。

\begin{figure}
\centering
\includegraphics[width=0.6\textwidth]{figure/outliers1}
\caption{Refined Part Pooling示意图}
\label{fig:refined}
\end{figure}

\subsection{监控摄像头部署方案的评价指标}

监控摄像头的部署方案包括摄像头的数量以及部署位置。摄像头的部署位置会影响监控场景的完整性、监控画面的光线质量以及监控目标的呈现角度。而监控摄像头数量受成本预算的限制,不可能无限增加,因此在预算有限的约束下,如何设计监控摄像头的部署位置,使得监控效果最优,便成为一个值得研究的问题。而在研究优化问题之前,需要定义监控摄像头部署方案的评价指标。

监控效果的优劣可以定义为在当前的监控方案下,跨摄像头追踪特定行人的能力。当前在多摄像头多行人追踪(Multi-Target Multi-Camera Tracking,Multi-Target Multi-Camera Tracking)领域主流的评价指标有多目标跟踪准确度($\mathit{MOTA}$)和识别F值($\mathit{IDF_1}$)。

\begin{figure}
\centering
\includegraphics[width=1\textwidth]{figure/structure2}
\caption{PCB+RPP架构图}
\label{fig:structure2}
\end{figure}

\subsubsection{多目标跟踪准确度($\mathit{MOTA}$)}

多目标跟踪准确度(Multiple Object Tracking Accuracy,$\mathit{MOTA}$)是衡量多目标追踪效果的常见指标。对于一段视频,其画面帧的总数为$T$,那么该段视频的多目标追踪准确度($\mathit{MOTA}$)的定义为:
\begin{equation}
\mathit{MOTA}=1-\frac{\mathit{FP}+\mathit{FN}+\Phi}{T}
\end{equation}
其中$\mathit{FN}$是视频所有帧中将负样本预测为正的总数,$\mathit{FP}$是视频所有帧中将正样本预测为负的总数,$\Phi$是预测序列中标签跳变的次数。MOTA指标的值域为$(-\infty,1]$,越接近1代表跟踪的效果越好。

\subsubsection{识别F值($\mathit{IDF_1}$)}

相比与$\mathit{MOTA}$中更多地关注目标人群的召回率以及目标追踪的稳定性,识别F值(Identification F-Score,$\mathit{IDF_1}$)更关注多摄像头多行人追踪过程中,行人标签的准确率。对于一段总帧数为$T$的视频,其识别F值($\mathit{IDF_1}$)定义为:
\begin{equation}
\mathit{IDF_1}=\frac{2\times\mathit{IDTP}}{2\times\mathit{IDTP}+\mathit{IDFP}+\mathit{IDFN}}
\end{equation}
其中$\mathit{IDTP}$是视频所有帧中将人物标签预测准确的总和,$\mathit{IDFP}$是视频所有帧中将正样本的人物标签预测错误的总和,$\mathit{IDFN}$是视频所有帧中将负样本的人物标签预测错误的总和。

\subsubsection{评价指标与当前的数据库结合}

结合本项目的实际情况,以及第一次预拍摄收集到的数据集的特点,将原始的17个摄像头中的第1个作为行人重识别算法的图库图片来源,其余16个摄像头按照物理位置和拍摄画面分为$G=5$组,每组摄像头个数为2至4个不等,定义$N=16$为待分配的摄像头总数,第$i$组的摄像头个数为$C_i$,同一组内摄像头大致拍摄到同一个物理位置,代表人物监控追踪场景中重点关注的位置。第$i$组的第$j$个摄像头定义为$c_{i,j}$,摄像头$c_{i,j}$的画面帧集合为$I_{c_{i,j}}$。假设在预算有限的前提下,每个位置(每个分组)只能选择1个摄像头,如何从组内选择合适的摄像头,使得监控效果最佳,便是需要解决的问题。该优化问题可以用数学语言形式化表示为:
\begin{equation}
\max \left\{\mathop{\rm eval}\left(\sum_{i=1}^G I_{c_{i,j}}\right)\,\middle\vert\, 1\leq i \leq G, 1\leq j \leq C_i\right\}
\end{equation}
其中$I_1+I_2$表示摄像头1的视频数据与摄像头2的视频数据在时序上依次拼接。$\mathop{\rm eval}(I)$表示对视频数据$I$做多目标跟踪准确度($\mathit{MOTA}$)或识别F值($\mathit{IDF_1}$)评估。

\subsection{强化学习模型}

强化学习(Reinforcement Learning)是近年来十分流行的人工智能算法,相比于监督学习(Supervised Learning),强化学习不需要整个环境(Environment)所有情况的监督信息,只需要环境在某种特定的情况下给出相应的反馈(Reward)。在很多现实问题当中,优化空间的状态个数可能是个非常大的数字,且很证明是否收集到足够多样本,可以用来近似代表位置环境状态的分布。同时,在围棋问题中\cite{silver2016mastering},各状态空间很难用监督的方法给每个状态评估价值,而强化学习只需要环境在每次动作(Action)之后给出相应的反馈,即可逐渐向更优的方向前进。强化学习也不属于无监督学习(Unsupervised Learning),无监督学习对于出现在测试集却不在训练集的样本没有处理能力,只能错误地分到已有的类中,而强化学习可以应对没有遇到的情况。

强化学习模型中的一般形式是一个智能体(Agent)在一个客观的环境(Environment)中,每一个时刻处于一个状态(State),当前状态存在一个短期价值和长期价值,短期价值可以是采取某种动作(Action)之后得到的反馈(Reward),长期价值(Value)则表示当前状态到最终状态能够得到的所有反馈总和的最大值。智能体根据当前状态的短期或长期价值和某种策略(Policy)采取某种动作,可立刻得到得到环境的反馈(Reward),并据此按照状态转移规则转移到下一个状态。在本项目中,根据要解决的具体问题,将上述概念相应定义如下:

当前的状态的集合$S=\left\{\boldsymbol{s}\,\middle\vert\,\boldsymbol{s}\in\mathbb{R}^{N}, s_i\in\{0, 1\}\right\}$,其中$s_i=1$表示选择了第$i$个摄像头。智能体可以采取的动作集合为$A=\left\{\boldsymbol{a}\,\middle\vert\,\boldsymbol{a}\in\mathbb{R}^N,\boldsymbol{a}_i \in \{0,1\}\right\}$,按照策略 $P$ 来选择动作, 1 表示选择该摄像头,0表示不选该摄像头。在本项目中,进行了动作之后, 状态是确定的,不存在一个动作可能导致几个不同的后续状态的情况,即状态转移概率 $\pi\left(\boldsymbol{s}^t\,\middle\vert\,\boldsymbol{s}^{t-1}, \boldsymbol{a}\right)\equiv1$。执行动作之后得到的反馈$r=\mathop{\rm{eval}}\left(I^{(t+1)}\right)-\mathop{\rm{eval}}\left(I^{(t)}\right)$,其中$I^{(t)}$表示$t$时刻的视频数据,由$t$时刻的状态$\boldsymbol{s}^{(t)}$决定。当前状态 $S$ 的长期价值$V:S\to \mathbb{R}^N$,是一个可学习的变量,代表智能体对于当前环境的认识程度。智能体应对当前状态的策略$P:V \to A$,是长期价值 $V$ 到动作 $A$ 的映射,可以简单地用贪心的策略,即选择概率最高的 5 个摄像头。也可以用 Policy Network ,即深度神经网络来实现。在本项目中采用简单的贪心策略实现。

按照以上定义,智能体的一个动作就是选择一个合法的摄像头部署方案。对于一个动作,它的反馈就是下一个状态的性能指标($\mathit{MOTA}$ 或 $\mathit{IDF_1}$)减去当前状态的性能指标。在反馈已知的前提下,适合用 Q-Learning\cite{watkins1989learning} 算法。在Q-Learning算法中,智能体关于当前状态的长期价值表示为一个价值矩阵$Q$。Q-Learning算法首先随机初始化各个状态的长期价值,让智能体在环境中随机游走,每走一步会得到一个反馈,根据反馈更新当前状态的长期价值,直到收敛。智能体从而可学习出一个对于该环境的认知。本项目中使用Q-Learning算法求当前状态的长期价值如算法\ref{alg:qlearning}所示,其中最关键的算法在于如何更新长期价值$Q$,更新的公式中包含两个参数学习率$\alpha$和远见性$\gamma$,学习率$\alpha$表示$Q$的更新速度,远见性$\gamma$的取值范围为$(0, 1)$,表示智能体对于当前反馈与长远价值的重视程度,$\gamma$越大,代表越重视长远价值。

\begin{algorithm}
    \caption{Q-Learning 算法求当前状态的长期价值}
    \label{alg:qlearning}
    \begin{algorithmic}[1]
        \Require 状态集合$S$、动作集合$A$、生命周期数$N$、学习率$\alpha$、远见性$\gamma$
        \Ensure 长期价值矩阵$Q$
        \Function {QLearning}{$S, A, N, \alpha, \gamma$}
            \State 随机初始化长期价值矩阵$Q$
            \For{$i=0\to N$}
                \State $\boldsymbol{s}^{(0)}\gets\boldsymbol{s}^\star$,$\boldsymbol{s}^\star$为从状态集合$S$中随机初始化的智能体的状态
                \State $t\gets1$
                \Repeat
                    \State $\boldsymbol{a}^{(t)}\gets \boldsymbol{a}^\star$,$\boldsymbol{a}^\star$为从动作集合$A$中随机选取的动作
                    \State $\boldsymbol{s}^{(t+1)}\gets \pi\left(\boldsymbol{s}^{(t)}, \boldsymbol{a}^{(t)}\right)$
                    \State $r^{(t)}\gets R\left(\boldsymbol{s}^{(t+1)},\boldsymbol{s}^{(t)}\right)$
                    \State $Q\left(\boldsymbol{s}^{(t)},\boldsymbol{a}^{(t)}\right)\gets(1-\alpha)\times Q\left(\boldsymbol{s}^{(t)},\boldsymbol{a}^{(t)}\right)+\alpha\times\left[r^{(t)}+\gamma\times\max_{\boldsymbol{a}'}Q\left(\boldsymbol{s}^{(t+1)}, \boldsymbol{a}'\right)\right]$
                    \State $\boldsymbol{s}^{(t)}\gets\boldsymbol{s}^{(t+1)}$
                    \State $t\gets  t+1$
                \Until{$\boldsymbol{s}^{(t)}=\boldsymbol{s}^{(0)}$}
            \EndFor
            \State\Return $Q$
        \EndFunction
    \end{algorithmic}
\end{algorithm}

\subsection{面向CPU集群的分布式深度学习训练框架}

\begin{figure}
\centering
\includegraphics[width=0.7\textwidth]{figure/dist}
\caption{分布式神经网络训练架构图}
\label{fig:dist}
\end{figure}

计算机集群是一种计算机组织的物理形态,它是由一组彼此连接的计算机组成的,这些计算机一般协同完成同一项计算任务。在同一集群中每台计算机的内部结构可以不同,特别地,若集群中每台计算机主要的算力提供者是CPU,那么称该集群为CPU集群。分布式计算是计算的一种工作方式,它将一个计算任务划分成多个子任务,每一个子任务与其它子任务相对独立,可以在时间上并行计算,以缩短计算时间。CPU集群提供了一组在物理上相对独立的计算机,所以可以很自然地考虑将分布式计算中的各个子任务部署到CPU集群中,充分利用CPU集群的计算资源。

深度神经网络模型的训练过程一般可分为前向计算(Forward)、误差反向传播(Loss Backpropagation)和参数更新。其中前向计算和反向传播的计算量很大,而且可以针对不同的训练集进行同步计算,因此深度神经网络模型的训练过程在结构上很适合进行分布式训练。在本项目中,集群的数量$n=5$,每个节点属于天河二号的GPU分区,具备高性能的CPU和GPU计算资源,节点之间通过千兆网络连接,避免各节点的通信速度成为分布式计算的性能瓶颈。

分布式训练架构如图\ref{fig:dist}所示,将训练数据通过随机采样的方式平均分成$n$份,分别输入集群中的各计算机。每一台计算机内存中包含一个独立的深度神经网络模型,进行该份训练数据的前向计算和反向传播计算,得到该批次训练数据在当前模型下的各参数梯度。参数服务器的计算任务是收集集群中各计算机回传的梯度,更新模型参数,并将新参数分发给各计算机,各计算机得到新参数后进行下一批次训练数据的计算。更新模型参数的方法为:
\begin{eqnarray}
\Delta W=\frac{1}{n}\sum_{i = 1}^{n}\Delta W_i \\
W=W-\eta\Delta W
\end{eqnarray}
其中$\eta$是参数的学习率。